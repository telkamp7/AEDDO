\documentclass[aspectratio=169]{beamer}

% Default packages
\usepackage[T1]{fontenc}
\usepackage[english]{babel}
\usepackage{pgfplots}
\pgfplotsset{compat=newest}
\usepackage{booktabs}
\usepackage{siunitx}
\usepackage{amsmath}

% Font selection
% Latin Modern
\usepackage{lmodern}
% Verdana font type
%\usepackage{verdana}
% Helvetica
%\usepackage{helvet}
% Times (text and math)
%\usepackage{newtx, newtxmath}
% Nice font combination
%\usepackage{mathptmx} % math
%\usepackage{sourcesanspro} % sans-serif
\usepackage{charter} % serif

%Avoid shaded in RMarkdown
\usepackage{color}
\usepackage{fancyvrb}
\newcommand{\VerbBar}{|}
\newcommand{\VERB}{\Verb[commandchars=\\\{\}]}
\DefineVerbatimEnvironment{Highlighting}{Verbatim}{commandchars=\\\{\}}
% Add ',fontsize=\small' for more characters per line
\usepackage{framed}
\definecolor{shadecolor}{RGB}{248,248,248}
\newenvironment{Shaded}{\begin{snugshade}}{\end{snugshade}}
\newcommand{\AlertTok}[1]{\textcolor[rgb]{0.94,0.16,0.16}{#1}}
\newcommand{\AnnotationTok}[1]{\textcolor[rgb]{0.56,0.35,0.01}{\textbf{\textit{#1}}}}
\newcommand{\AttributeTok}[1]{\textcolor[rgb]{0.77,0.63,0.00}{#1}}
\newcommand{\BaseNTok}[1]{\textcolor[rgb]{0.00,0.00,0.81}{#1}}
\newcommand{\BuiltInTok}[1]{#1}
\newcommand{\CharTok}[1]{\textcolor[rgb]{0.31,0.60,0.02}{#1}}
\newcommand{\CommentTok}[1]{\textcolor[rgb]{0.56,0.35,0.01}{\textit{#1}}}
\newcommand{\CommentVarTok}[1]{\textcolor[rgb]{0.56,0.35,0.01}{\textbf{\textit{#1}}}}
\newcommand{\ConstantTok}[1]{\textcolor[rgb]{0.00,0.00,0.00}{#1}}
\newcommand{\ControlFlowTok}[1]{\textcolor[rgb]{0.13,0.29,0.53}{\textbf{#1}}}
\newcommand{\DataTypeTok}[1]{\textcolor[rgb]{0.13,0.29,0.53}{#1}}
\newcommand{\DecValTok}[1]{\textcolor[rgb]{0.00,0.00,0.81}{#1}}
\newcommand{\DocumentationTok}[1]{\textcolor[rgb]{0.56,0.35,0.01}{\textbf{\textit{#1}}}}
\newcommand{\ErrorTok}[1]{\textcolor[rgb]{0.64,0.00,0.00}{\textbf{#1}}}
\newcommand{\ExtensionTok}[1]{#1}
\newcommand{\FloatTok}[1]{\textcolor[rgb]{0.00,0.00,0.81}{#1}}
\newcommand{\FunctionTok}[1]{\textcolor[rgb]{0.00,0.00,0.00}{#1}}
\newcommand{\ImportTok}[1]{#1}
\newcommand{\InformationTok}[1]{\textcolor[rgb]{0.56,0.35,0.01}{\textbf{\textit{#1}}}}
\newcommand{\KeywordTok}[1]{\textcolor[rgb]{0.13,0.29,0.53}{\textbf{#1}}}
\newcommand{\NormalTok}[1]{#1}
\newcommand{\OperatorTok}[1]{\textcolor[rgb]{0.81,0.36,0.00}{\textbf{#1}}}
\newcommand{\OtherTok}[1]{\textcolor[rgb]{0.56,0.35,0.01}{#1}}
\newcommand{\PreprocessorTok}[1]{\textcolor[rgb]{0.56,0.35,0.01}{\textit{#1}}}
\newcommand{\RegionMarkerTok}[1]{#1}
\newcommand{\SpecialCharTok}[1]{\textcolor[rgb]{0.00,0.00,0.00}{#1}}
\newcommand{\SpecialStringTok}[1]{\textcolor[rgb]{0.31,0.60,0.02}{#1}}
\newcommand{\StringTok}[1]{\textcolor[rgb]{0.31,0.60,0.02}{#1}}
\newcommand{\VariableTok}[1]{\textcolor[rgb]{0.00,0.00,0.00}{#1}}
\newcommand{\VerbatimStringTok}[1]{\textcolor[rgb]{0.31,0.60,0.02}{#1}}
\newcommand{\WarningTok}[1]{\textcolor[rgb]{0.56,0.35,0.01}{\textbf{\textit{#1}}}}



% Use DTU theme, see below for options
\usetheme[department=compute]{DTU}

\title[Automated and Early Detection of Disease Outbreaks]{Models}
\author{Kasper Schou Telkamp}
\institute{Section for Dynamical Systems}
\date{2023-02-09}
	
\newcommand{\tabitem}{{\color{dtured}$\bullet$} }

\DeclareMathOperator{\E}{E}
\DeclareMathOperator{\V}{V}
\DeclareMathOperator{\G}{G}
\DeclareMathOperator{\N}{N}
\DeclareMathOperator{\NB}{NB}
\DeclareMathOperator{\Pois}{Pois}
\DeclareMathOperator{\Geom}{Geom}


\begin{document}


\frame{
	\maketitle
}

\frame{
	\frametitle{Outline}
	\tableofcontents
}

\hypertarget{data-exploration}{%
\section{Data exploration}\label{data-exploration}}

\hypertarget{vtec-stec}{%
\subsection*{VTEC / STEC}\label{vtec-stec}}

\begin{frame}{VTEC / STEC}
\tiny

\begin{table}
\centering\begingroup\fontsize{12}{14}\selectfont

\begin{tabular}{llll}
\toprule
Date & ageGroup & $y_{it}$ & $x_{it}$\\
\midrule
2008-01-01 & <1 year & 2 & 64137\\
2008-01-01 & 1-4 years & 2 & 259910\\
2008-01-01 & 5-14 years & 2 & 680529\\
2008-01-01 & 15-24 years & 1 & 635838\\
... & ... & ... & ...\\
2022-12-01 & 5-14 years & 5 & 634139\\
2022-12-01 & 15-24 years & 1 & 721286\\
2022-12-01 & 25-64 years & 10 & 3031374\\
2022-12-01 & 65+ years & 12 & 1204892\\
\bottomrule
\end{tabular}
\endgroup{}
\end{table}

\normalsize
\end{frame}

\begin{frame}{VTEC / STEC}
\protect\hypertarget{vtec-stec-1}{}
\tiny

\includegraphics[width=1\linewidth]{../figures/EpixSTEC}

\normalsize
\end{frame}

\begin{frame}{VTEC / STEC}
\protect\hypertarget{vtec-stec-2}{}
\tiny

\includegraphics[width=1\linewidth]{../figures/STECxEpixAgeGroup}

\normalsize
\end{frame}

\begin{frame}{VTEC / STEC}
\protect\hypertarget{vtec-stec-3}{}
\tiny

\includegraphics[width=1\linewidth]{../figures/ShigaogveratoxinproducerendeEcolixAgeGroup}

\normalsize
\end{frame}

\hypertarget{hierachical-poisson-normal-model}{%
\section{Hierachical Poisson-Normal
model}\label{hierachical-poisson-normal-model}}

\hypertarget{formulation}{%
\subsection{Formulation}\label{formulation}}

\begin{frame}{Formulation}
The conditional distribution, \(Y|u\), of the count observations are
assumed to be a Poisson distribution with intensities \(\lambda_{it}\).
Also, we shall assume that the count is proportional to the population
size, \(x_{it}\), within each age group, \(i\), at a given time point,
\(t\). Hence, in terms of the canonical link for the Poisson
distribution the model for the fixed effect is

\begin{equation}
  \log(\lambda_{it})=\mathbf{X}_i^T\mathbf{\beta}_{it}+\log(x_{it})
\end{equation}

Here \(\mathbf{X}_i\) is \(T\times6\)-dimensional, and
\(\mathbf{\beta}_{it}\) contains the corresponding fixed effect
parameter. The random effects \(u_{it}\) are assumed to be Gaussian.

\begin{equation}
  u_{it} = \epsilon_{it}
\end{equation}

where \(\epsilon_{it}\sim\N(0,\sigma^2)\) is a white noise process, and
\(\sigma\) is a model parameter.
\end{frame}

\begin{frame}{Fomulation}
\protect\hypertarget{fomulation}{}
Henceforth, the model can be formulated as a two-level hierarchical
model

\begin{subequations}
  \begin{alignat}{2}
    Y_{it}|u_{it} &\sim \Pois (\lambda_{it}e^{u_{it}}) \label{eq:pois_ln0} \\ 
    u_{it} &\sim \N(0,\sigma^2) \label{eq:pois_ln1}
  \end{alignat}
\end{subequations}
\end{frame}

\hypertarget{implementation}{%
\subsection{Implementation}\label{implementation}}

\begin{frame}[fragile]{Implementation - Objective function in C++}
\protect\hypertarget{implementation---objective-function-in-c}{}
\tiny

\begin{Shaded}
\begin{Highlighting}[]
\PreprocessorTok{\#include }\ImportTok{\textless{}TMB.hpp\textgreater{}}\PreprocessorTok{              }\CommentTok{// Links in the TMB libraries}

\KeywordTok{template}\OperatorTok{\textless{}}\KeywordTok{class}\NormalTok{ Type}\OperatorTok{\textgreater{}}
\NormalTok{Type objective\_function}\OperatorTok{\textless{}}\NormalTok{Type}\OperatorTok{\textgreater{}::}\KeywordTok{operator}\OperatorTok{()} \OperatorTok{()}
\OperatorTok{\{}
\NormalTok{  DATA\_VECTOR}\OperatorTok{(}\NormalTok{y}\OperatorTok{);}                               \CommentTok{// Data vector transmitted from R}
\NormalTok{  DATA\_VECTOR}\OperatorTok{(}\NormalTok{x}\OperatorTok{);}                       \CommentTok{// Data vector transmitted from R}
\NormalTok{  DATA\_MATRIX}\OperatorTok{(}\NormalTok{X}\OperatorTok{);}                       \CommentTok{// Design matrix transmitted from R}
  
\NormalTok{  PARAMETER\_VECTOR}\OperatorTok{(}\NormalTok{u}\OperatorTok{);}                      \CommentTok{// Random effects}
  
  \CommentTok{// Parameters}
\NormalTok{  PARAMETER\_VECTOR}\OperatorTok{(}\NormalTok{beta}\OperatorTok{);}         \CommentTok{// Parameter value transmitted from R}
\NormalTok{  PARAMETER}\OperatorTok{(}\NormalTok{log\_sigma\_u}\OperatorTok{);}               \CommentTok{// Parameter value transmitted from R}
  
\NormalTok{  vector}\OperatorTok{\textless{}}\NormalTok{Type}\OperatorTok{\textgreater{}}\NormalTok{ lambda  }\OperatorTok{=}\NormalTok{ exp}\OperatorTok{(}\NormalTok{X}\OperatorTok{*}\NormalTok{beta}\OperatorTok{{-}}\NormalTok{log}\OperatorTok{(}\NormalTok{x}\OperatorTok{)+}\NormalTok{u}\OperatorTok{);}
\NormalTok{  Type sigma\_u }\OperatorTok{=}\NormalTok{ exp}\OperatorTok{(}\NormalTok{log\_sigma\_u}\OperatorTok{);}
  
  \DataTypeTok{int}\NormalTok{ nobs }\OperatorTok{=}\NormalTok{ y}\OperatorTok{.}\NormalTok{size}\OperatorTok{();}
\NormalTok{  Type mean\_ran }\OperatorTok{=}\NormalTok{ Type}\OperatorTok{(}\DecValTok{0}\OperatorTok{);}
  
  \DataTypeTok{int}\NormalTok{ i}\OperatorTok{;}
  
\NormalTok{  Type f }\OperatorTok{=} \DecValTok{0}\OperatorTok{;}                           \CommentTok{// Declare the "objective function"}
  \ControlFlowTok{for}\OperatorTok{(}\DataTypeTok{int}\NormalTok{ t}\OperatorTok{=}\DecValTok{0}\OperatorTok{;}\NormalTok{ t }\OperatorTok{\textless{}}\NormalTok{ nobs}\OperatorTok{;}\NormalTok{ t}\OperatorTok{++)\{}
\NormalTok{    f }\OperatorTok{{-}=}\NormalTok{ dnorm}\OperatorTok{(}\NormalTok{u}\OperatorTok{[}\NormalTok{t}\OperatorTok{],}\NormalTok{mean\_ran}\OperatorTok{,}\NormalTok{sigma\_u}\OperatorTok{,}\KeywordTok{true}\OperatorTok{);}
\NormalTok{    f }\OperatorTok{{-}=}\NormalTok{ dpois}\OperatorTok{(}\NormalTok{y}\OperatorTok{[}\NormalTok{t}\OperatorTok{],}\NormalTok{lambda}\OperatorTok{[}\NormalTok{t}\OperatorTok{],}\KeywordTok{true}\OperatorTok{);}
  \OperatorTok{\}}
  
  \ControlFlowTok{return}\NormalTok{ f}\OperatorTok{;}
\OperatorTok{\}}
\end{Highlighting}
\end{Shaded}

\normalsize
\end{frame}

\begin{frame}[fragile]{Implementation - Call from R}
\protect\hypertarget{implementation---call-from-r}{}
\tiny

\begin{Shaded}
\begin{Highlighting}[]
\CommentTok{\# Import libraries}
\FunctionTok{library}\NormalTok{(readr); }\FunctionTok{library}\NormalTok{(dplyr); }\FunctionTok{library}\NormalTok{(TMB)}

\CommentTok{\# Import the data}
\NormalTok{dat }\OtherTok{\textless{}{-}} \FunctionTok{read\_rds}\NormalTok{(}\AttributeTok{file =} \StringTok{"../../data/processed/dat.rds"}\NormalTok{)}

\CommentTok{\# Only consider some of the data}
\NormalTok{y }\OtherTok{\textless{}{-}}\NormalTok{ dat }\SpecialCharTok{\%\textgreater{}\%}
  \FunctionTok{filter}\NormalTok{(caseDef }\SpecialCharTok{==} \StringTok{"Shiga{-} og veratoxin producerende E. coli."}\NormalTok{) }\SpecialCharTok{\%\textgreater{}\%}
  \FunctionTok{group\_by}\NormalTok{(Date, ageGroup) }\SpecialCharTok{\%\textgreater{}\%} \FunctionTok{reframe}\NormalTok{(}\AttributeTok{y =} \FunctionTok{sum}\NormalTok{(cases), }\AttributeTok{n =} \FunctionTok{sum}\NormalTok{(n))}

\FunctionTok{compile}\NormalTok{(}\AttributeTok{file =} \StringTok{"PoissonNormal.cpp"}\NormalTok{)  }\CommentTok{\# Compile the C++ file}
\FunctionTok{dyn.load}\NormalTok{(}\FunctionTok{dynlib}\NormalTok{(}\StringTok{"PoissonNormal"}\NormalTok{))    }\CommentTok{\# Dynamically link the C++ code}

\CommentTok{\# Make months into integers}
\NormalTok{y.design }\OtherTok{\textless{}{-}}\NormalTok{ y }\SpecialCharTok{\%\textgreater{}\%} \FunctionTok{mutate}\NormalTok{(}\AttributeTok{Month =} \FunctionTok{as.integer}\NormalTok{(}\FunctionTok{format}\NormalTok{(Date, }\StringTok{"\%m"}\NormalTok{)))}

\CommentTok{\# Construct the design matrix}
\NormalTok{designMatrix }\OtherTok{\textless{}{-}} \FunctionTok{model.matrix}\NormalTok{(y }\SpecialCharTok{\textasciitilde{}} \SpecialCharTok{{-}}\DecValTok{1} \SpecialCharTok{+}\NormalTok{ ageGroup, }\AttributeTok{data =}\NormalTok{ y.design)}

\CommentTok{\# Construct initial parameters}
\NormalTok{beta }\OtherTok{\textless{}{-}} \FunctionTok{rep}\NormalTok{(}\DecValTok{1}\NormalTok{, }\FunctionTok{nlevels}\NormalTok{(y}\SpecialCharTok{$}\NormalTok{ageGroup))}

\CommentTok{\# Function and derivative}
\NormalTok{PoisLN }\OtherTok{\textless{}{-}} \FunctionTok{MakeADFun}\NormalTok{(}
  \AttributeTok{data =} \FunctionTok{list}\NormalTok{(}\AttributeTok{y =}\NormalTok{ y}\SpecialCharTok{$}\NormalTok{y, }\AttributeTok{x =}\NormalTok{ y}\SpecialCharTok{$}\NormalTok{n, }\AttributeTok{X =}\NormalTok{ designMatrix),}
  \AttributeTok{parameters =} \FunctionTok{list}\NormalTok{(}\AttributeTok{u =} \FunctionTok{rep}\NormalTok{(}\DecValTok{1}\NormalTok{, }\FunctionTok{length}\NormalTok{(y}\SpecialCharTok{$}\NormalTok{y)), }\AttributeTok{beta =}\NormalTok{ beta, }\AttributeTok{log\_sigma\_u =} \FunctionTok{log}\NormalTok{(}\DecValTok{1}\NormalTok{)),}
  \AttributeTok{random =} \StringTok{"u"}\NormalTok{, }\AttributeTok{DLL =} \StringTok{"PoissonNormal"}\NormalTok{)}

\NormalTok{opt }\OtherTok{\textless{}{-}} \FunctionTok{nlminb}\NormalTok{(}\AttributeTok{start =}\NormalTok{ PoisLN}\SpecialCharTok{$}\NormalTok{par, PoisLN}\SpecialCharTok{$}\NormalTok{fn, PoisLN}\SpecialCharTok{$}\NormalTok{gr, }\AttributeTok{lower =} \FunctionTok{c}\NormalTok{(}\FloatTok{0.01}\NormalTok{, }\FloatTok{0.01}\NormalTok{))}
\end{Highlighting}
\end{Shaded}

\normalsize
\end{frame}

\hypertarget{results}{%
\subsection{Results}\label{results}}

\begin{frame}{Results}
\tiny

\begin{table}
\centering\begingroup\fontsize{10}{12}\selectfont

\begin{tabular}{lrr}
\toprule
Parameter & Estimate & Std. Error\\
\midrule
$\beta_{<1 year}$ & 10.83 & 0.11\\
$\beta_{1-4 years}$ & 13.64 & 0.09\\
$\beta_{5-14 years}$ & 13.94 & 0.10\\
$\beta_{15-24 years}$ & 14.05 & 0.09\\
$\beta_{25-64 years}$ & 16.61 & 0.08\\
$\beta_{65+ years}$ & 14.83 & 0.09\\
$\sigma$ & 1.01 & 0.04\\
\bottomrule
\end{tabular}
\endgroup{}
\end{table}

\normalsize
\end{frame}

\begin{frame}{Results}
\protect\hypertarget{results-1}{}
\tiny

\includegraphics[width=1\linewidth]{../figures/windowedSTEDPoisN}

\normalsize
\end{frame}

\begin{frame}{Results}
\protect\hypertarget{results-2}{}
\tiny

\includegraphics[width=1\linewidth]{../figures/windowedSTEDPoisNExclude}

\normalsize
\end{frame}

\hypertarget{hierachical-poisson-gamma-model}{%
\section{Hierachical Poisson-Gamma
model}\label{hierachical-poisson-gamma-model}}

\hypertarget{formulation-1}{%
\subsection{Formulation}\label{formulation-1}}

\begin{frame}{Formulation}
Likewise, in the compound Poisson-Gamma model the conditional
distribution, \(Y|u\), of the count observations are assumed to be a
Poisson distribution, but this time the intensities, \(\lambda_{it}\),
are defined as

\begin{equation}
  \log(\lambda_{it})=\mathbf{X}_i^T\mathbf{\beta}_{it}+\log(x_{it})
\end{equation}

Here \(\mathbf{X}_i\) is \(T\times6\)-dimensional, and
\(\mathbf{\beta}_{it}\) contains the corresponding fixed effect
parameter. Additionally, the random effects \(u_{it}\) are assumed to be
Gamma distributed.
\end{frame}

\begin{frame}{Formulation}
\protect\hypertarget{formulation-2}{}
Subsequently, the model can be formulated as a two-level hierarchical
model

\begin{subequations} \label{eq:PoisGam}
  \begin{alignat}{2}
    Y_{it}|u_{it} &\sim \Pois (\lambda_{it}u_{it}) \label{eq:pois_g0} \\ 
    u_{it} &\sim \G(1/\phi_{i},\phi_{i}) \label{eq:pois_g1}
  \end{alignat}
\end{subequations}
\end{frame}

\hypertarget{probability-function-for-y}{%
\subsection{\texorpdfstring{Probability function for
\(Y\)}{Probability function for Y}}\label{probability-function-for-y}}

\begin{frame}{Probability function for \(Y\)}
\begin{equation} \label{eq:pdfMix}
  \begin{aligned}
    P[Y=y_i]&=g_{Y}(y;\lambda, \phi) \\
    &=\frac{\lambda^{y}}{y!\Gamma(1/\phi)\phi^{1/\phi}}\frac{\phi^{y+1/\phi}\Gamma(y+1/\phi)}{(\lambda \phi + 1)^{y+1/\phi}} \\
    &=\frac{\Gamma(y+1/\phi)}{\Gamma(1/\phi)y!}\frac{1}{(\lambda\phi+1)^{1/\phi}}\bigg(\frac{\lambda\phi}{\lambda\phi+1}\bigg)^{y} \\
    &=\begin{pmatrix} y+1/\phi-1 \\ y \end{pmatrix} \frac{1}{(\lambda\phi+1)^{1/\phi}}\bigg(\frac{\lambda\phi}{\lambda\phi+1}\bigg)^{y} \ , \ for \ y = 0, 1, 2, \dots
  \end{aligned}
\end{equation}

where we have used the convention

\begin{equation}
  \begin{pmatrix} z\\y \end{pmatrix} = \frac{\Gamma(z+1)}{\Gamma(z+1-y)y!}
\end{equation}

The marginal distribution of \(Y\) is a negative binomial distribution,
\(Y\sim \NB\big(1/\phi,1/(\lambda \phi+1)\big)\)
\end{frame}

\begin{frame}{Proof}
\protect\hypertarget{proof}{}
The probability function for the conditional distribution of \(Y\) for
given \(u\)

\begin{equation} \label{eq:pdfPois}
  f_{Y|u}(y;\lambda, u)=\frac{(\lambda u)^y}{y!} \exp (-\lambda u)
\end{equation}

and the probability density function for the distribution of \(u\) is

\begin{equation} \label{eq:pdfGamma}
  f_{u}(u;\phi)=\frac{1}{\phi \Gamma(1/\phi)} \bigg(\frac{u}{\phi}\bigg)^{1/\phi-1} \exp (-u/\phi)
\end{equation}
\end{frame}

\begin{frame}{Proof}
\protect\hypertarget{proof-1}{}
Given \eqref{eq:pdfPois} and \eqref{eq:pdfGamma}, the probability
function for the marginal distribution of \(Y\) is determined from

\begin{equation} \label{eq:marMix}
  \begin{aligned}
    g_{Y}(y;\lambda,\phi)&=\int_{u=0}^\infty f_{Y|u}(y;\lambda, u) f_{u}(u;\phi) \,du \\
    &=\int_{u=0}^\infty \frac{(\lambda u)^y}{y!} \exp (-\lambda u) \frac{1}{\phi \Gamma(1/\phi)} \bigg(\frac{u}{\phi}\bigg)^{1/\phi-1} \exp (-u/\phi) \,du\\
    &=\frac{\lambda^{y}}{y!\Gamma(1/\phi)\phi^{1/\phi}} \int_{u=0}^\infty u^{y+1/\phi-1} \exp \big(-u(\lambda \phi+1)/\phi\big) \,du
  \end{aligned}
\end{equation}
\end{frame}

\begin{frame}{Proof}
\protect\hypertarget{proof-2}{}
In \eqref{eq:marMix} it is noted that the integrand is the \emph{kernel}
in the probability density function for a Gamma distribution,
\(\G\big(y+1/\phi,\phi/(\lambda \phi+1)\big)\). As the integral of the
density shall equal one, we find by adjusting the norming constant that

\begin{equation}
  \int_{u=0}^\infty u^{y+1/\phi-1} \exp \bigg(-u/\Big(\phi/(\lambda \phi+1)\Big)\bigg) \,du = \frac{\phi^{y+1/\phi}\Gamma(y+1/\phi)}{(\lambda \phi + 1)^{y+1/\phi}}
\end{equation}

and then \eqref{eq:pdfMix} follows
\end{frame}

\hypertarget{inference-on-individual-group-means}{%
\subsection{Inference on individual group
means}\label{inference-on-individual-group-means}}

\begin{frame}{Inference on individual group means}
Consider the hierarchical Poisson-Gamma model in \eqref{eq:PoisGam}, and
assume that a value \(Y=y\) has been observed. Then the conditional
distribution of \(u\) for given \(Y=y\) is a Gamma distribution,

\begin{equation}
  u|Y=y\sim \G\big(y+1/\phi,\phi/(\lambda \phi+1)\big)
\end{equation}

with mean

\begin{equation}
  \E[u|Y=y]=\frac{y\phi+1}{\lambda\phi+1}
\end{equation}

and variance

\begin{equation}
  \V[u|Y=y]=\frac{(y \phi^2+\phi)}{(\lambda \phi + 1)^2}
\end{equation}
\end{frame}

\begin{frame}{Proof}
\protect\hypertarget{proof-3}{}
The conditional distribution is found using Bayes Theorem

\begin{equation}
  \begin{aligned}
    g_{u}(u|Y=y)&=\frac{f_{y,u}(y,u)}{g_Y(y;\lambda, \phi)} \\
    &=\frac{f_{y|u}(y;u)g_{u}(u)}{g_{Y}(y;\lambda,\phi)} \\
    &=\frac{1}{g_{Y}(y;\lambda,\phi)}\bigg(\frac{(\lambda u)^y}{y!} \exp (-\lambda u) \frac{1}{\phi \Gamma(1/\phi)} \bigg(\frac{u}{\phi}\bigg)^{1/\phi-1} \exp (-u/\phi)\bigg) \\
    &\propto u^{y+1/\phi-1} \exp \big(- u(\lambda\phi+1)/\phi\big)
  \end{aligned}
\end{equation}

We identify the \emph{kernel} of the probability density function

\begin{equation}
  u^{y+1/\phi-1} \exp (- u(\lambda\phi+1)/\phi)
\end{equation}

as the kernel of a Gamma distribution,
\(\G(y+1/\phi,\phi/(\lambda\phi+1))\)
\end{frame}

\hypertarget{implementation-1}{%
\subsection{Implementation}\label{implementation-1}}

\begin{frame}[fragile]{Implementation - Define negative likelihood
function in R}
\protect\hypertarget{implementation---define-negative-likelihood-function-in-r}{}
\tiny

\begin{Shaded}
\begin{Highlighting}[]
\CommentTok{\# Define the negative likelihood function for the marginal distribution of Y}
\NormalTok{nll.age }\OtherTok{\textless{}{-}} \ControlFlowTok{function}\NormalTok{(theta, data)\{}
  \CommentTok{\# Extract counts}
\NormalTok{  y }\OtherTok{\textless{}{-}}\NormalTok{ data}\SpecialCharTok{$}\NormalTok{y}
  \CommentTok{\# Extract agegroups}
\NormalTok{  ageGroup }\OtherTok{\textless{}{-}}\NormalTok{ data}\SpecialCharTok{$}\NormalTok{ageGroup}
  \CommentTok{\# Extract number of agegroups}
\NormalTok{  n.ageGroup }\OtherTok{\textless{}{-}} \FunctionTok{n\_distinct}\NormalTok{(data}\SpecialCharTok{$}\NormalTok{ageGroup)}
  
  \CommentTok{\# Define parameters}
\NormalTok{  lambda }\OtherTok{\textless{}{-}}\NormalTok{ theta[}\DecValTok{1}\SpecialCharTok{:}\NormalTok{n.ageGroup]}
\NormalTok{  beta }\OtherTok{\textless{}{-}}\NormalTok{ theta[(n.ageGroup}\SpecialCharTok{+}\DecValTok{1}\NormalTok{)}\SpecialCharTok{:}\NormalTok{(n.ageGroup}\SpecialCharTok{*}\DecValTok{2}\NormalTok{)]}
  
  \CommentTok{\# Construct the size and probability for the negative binomial distribution}
\NormalTok{  r }\OtherTok{\textless{}{-}} \DecValTok{1}\SpecialCharTok{/}\NormalTok{beta}
\NormalTok{  p }\OtherTok{\textless{}{-}} \DecValTok{1}\SpecialCharTok{/}\NormalTok{(lambda}\SpecialCharTok{*}\NormalTok{beta}\SpecialCharTok{+}\DecValTok{1}\NormalTok{)}
  
  \CommentTok{\# Initilize the log{-}likelihood}
\NormalTok{  ll }\OtherTok{\textless{}{-}} \DecValTok{0}
  \ControlFlowTok{for}\NormalTok{(i }\ControlFlowTok{in} \DecValTok{1}\SpecialCharTok{:}\FunctionTok{nrow}\NormalTok{(data))\{}
\NormalTok{    ll }\OtherTok{=}\NormalTok{ ll }\SpecialCharTok{+} \FunctionTok{dnbinom}\NormalTok{(}\AttributeTok{x =}\NormalTok{ y[i],}
                      \AttributeTok{size =}\NormalTok{ r[ageGroup[i]],}
                      \AttributeTok{prob =}\NormalTok{ p[ageGroup[i]],}
                      \AttributeTok{log =} \ConstantTok{TRUE}\NormalTok{) }
\NormalTok{  \}}
  
  \CommentTok{\# Return the negative log{-}likelihood}
  \SpecialCharTok{{-}}\NormalTok{ll}
\NormalTok{\}}
\end{Highlighting}
\end{Shaded}

\normalsize
\end{frame}

\hypertarget{results-3}{%
\subsection{Results}\label{results-3}}

\begin{frame}{Results}
\tiny

\begin{table}
\centering\begingroup\fontsize{10}{12}\selectfont

\begin{tabular}{lrr}
\toprule
Parameter & Estimate & Std. Error\\
\midrule
$\beta_{<1 year}$ & 11.23 & 0.08\\
$\beta_{1-4 years}$ & 13.94 & 0.06\\
$\beta_{5-14 years}$ & 14.30 & 0.07\\
$\beta_{15-24 years}$ & 14.42 & 0.07\\
$\beta_{25-64 years}$ & 16.87 & 0.06\\
$\beta_{65+ years}$ & 15.24 & 0.06\\
$\phi$ & 0.45 & 0.03\\
\bottomrule
\end{tabular}
\endgroup{}
\end{table}

\normalsize
\end{frame}

\begin{frame}{Results}
\protect\hypertarget{results-4}{}
\tiny

\includegraphics[width=1\linewidth]{../figures/windowedSTECPoisG}

\normalsize
\end{frame}

\begin{frame}{Results}
\protect\hypertarget{results-5}{}
\tiny

\includegraphics[width=1\linewidth]{../figures/windowedSTECPoisGExclude}

\normalsize
\end{frame}


\end{document}
